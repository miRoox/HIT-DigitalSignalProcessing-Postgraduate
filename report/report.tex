% !TEX encoding = UTF-8
\documentclass{hitgsrep}
\usepackage{graphicx}
\usepackage{amsmath}
\usepackage{physics}
\usepackage{tikz}
%\usepackage[outputdir={out}]{minted}

\usetikzlibrary{arrows,chains}

\hitgsrepset{
    author={卢}, % 学生姓名
    studentid={19SXXXXXX}, % 学号
    studentcat={硕士}, % 学生类别
    course={数字信号处理}, % 考核科目
    affiliation={仪器科学与工程学院}, % 学生所在院(系)
    discipline={仪器科学与技术}, % 学生所在学科
    year={2020}, % 年份(不填根据当前时间自动生成,一月时需要手动填入前一年的年份)
    semester={秋}, % 学期(不填根据当前时间自动生成)
}

\ctexset{
    section/format=\Large\bfseries
}

\newcommand{\subjectname}[1]{\begin{center}\LARGE\bfseries #1\end{center}}

\begin{document}
\maketitle

\subjectname{基于维纳反卷积对运动模糊图像复原}

\section{运动模糊图像的退化模型}

图像复原问题的关键在于退化模型的建立。
一般的,典型的图像退化模型如图~\ref{fig:model}

\begin{figure}[htb!]
    \centering
    \begin{tikzpicture}[
        font=\footnotesize,
        every node/.style={draw, text centered},
        every join/.style={->, draw},
    ]
    \tikzset{
        proc/.style={rectangle},
        sum/.style={circle},
    }
    \node[proc] {输入$f(x,y)$}; 
    \end{tikzpicture}
    \caption{图像退化模型}
    \label{fig:model}
\end{figure}

\end{document}
